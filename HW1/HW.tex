% !TEX program = xelatex
\documentclass[]{article}
\usepackage{commons/course}

\begin{document}
\printheader

در قسمت‌های الکترونیکی از توزیع نمایی یا
\lr{exponential}
استفاده می‌شود. دلیل استفاده از آن به خاطر ویژگی
\lr{memorylessness}
بودن آن است بدین معنا که
$\Pr(X>t+s \mid X>t)=\Pr(X>s)$
است. به زبان ساده‌تر سالم بودن یا خراب بودن دستگاه در گذشته تاثیری در تخمین کارکرد آن در آینده ندارد.

اما از طرفی این تعریف برای قطعات مکانیکی صادق نیست چرا که در قطعات مکانیکی به خاطر خوردگی هر چه قدر که زمان
به جلو پیش می‌رود و از عمر قطعه می‌گذرد بیشتر خورده می‌شود و در نتیجه قطعا خاصیت
\lr{memorylessness}
برای آن صادق نیست. به همین جهت برای احتمال خرابی قطعه به شرط گذشتن
$t$
واحد از تولید یا کارکرد آن از توزیع نرمال استفاده می‌کنند.

\end{document}
