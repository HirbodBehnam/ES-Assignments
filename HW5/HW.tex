% !TEX program = xelatex
\documentclass[]{article}
\usepackage{commons/course}

\begin{document}
\printheader
در این تمرین می‌خواهیم که درباره‌ی
\lr{Ambient Intelligence}
تحقیق کنیم که چیست و آینده آن به چه صورت است. برای این تحقیق من بیشتر از صفحه‌ی اصلی ویکیپدیا انگلیسی استفاده کردم.

در واقع
\lr{AmI}
عملا یک سری محیط‌هایی هستند که به کمک دستگاه‌های الکترونیکی به حضور و افراد حساس هستند و به آن واکنش نشان می‌دهند.
به عنوان مثال می‌توان به خانه‌های هوشمندی اشاره کرد به وارد شدن افراد به اتاق‌های مختلف چراق‌های آن
اتاق‌ها را روشن یا خاموش می‌کنند یا با توجه به حس افراد آهنگ مناسب پخش می‌کنند. به صورت کلی اینترنت اشیا یکی از مثال‌های
\lr{AmI}
است.

ویژگی‌هایی که ویکیپدیا برای اینکه یک سیستم جزو
\lr{AmI}
حساب شوند را به صورت زیر آورده است:
\begin{itemize}
    \item باید سیستم نهفته و فراوان باشند.
    \item باید به محیط اطرفشان مسلط باشند و افراد را شناسایی کنند.
    \item قابلیت شخصی سازی شدن را داشته باشند.
    \item با توجه به نیاز افراد باید تغییر پیدا کنند.
    \item به هوشمندانه تصمیم بگیرند.
\end{itemize}

به نظر من از آنجا که دستگاه‌های
\lr{CPS}
روز به روز در حال زیاد شدن هستند، پس به نظر من این دستگاه‌ها می‌توانند آینده خوبی داشته باشند چرا که نه تنها تعداد
آن‌ها در حال زیاد شدن است، بلکه قدرت پردازشی آن‌ها و
\lr{connectability}
آن‌ها نیز در حال افزایش است.
\end{document}
