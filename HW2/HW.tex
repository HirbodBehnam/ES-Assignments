% !TEX program = xelatex
\documentclass[]{article}
\usepackage{commons/course}

\begin{document}
\printheader

در این سوال می‌خواهیم که
\lr{FSM}
معرفی شده در کلاس را به کمک زبان پایتون شبیه سازی کنیم.
قسمت
\lr{super state}
آن در زیر آمده است.
\begin{latin}
    \centering
    \begin{tikzpicture}[->, >=stealth', auto, semithick, node distance=5cm]
    \tikzstyle{every state}=[fill=white,draw=black,thick,text=black,scale=1]
    \node[align=center,initial above,state,initial text=] (0)  {Heater Off\\Cooler Off};
    \node[align=center,state] (1)[right of=0] {Heater On\\Cooler Off};
    \node[align=center,state] (2)[left of=0]  {Heater Off\\Cooler On};
    \path
    (0) edge[bend left, above] node{$T < 15 \degree C$} (1)
    (1) edge[bend left, below] node{$T > 30 \degree C$} (0)
    (0) edge[bend left, below] node{$T > 35 \degree C$} (2)
    (2) edge[bend left, above] node{$T < 25 \degree C$} (0)
    ;
    \end{tikzpicture}
\end{latin}
هر کدام از حالاتی که گرم کن یا سرد کن روشن است یک زیر
\lr{state}
هستند. به عنوان مثال وقتی که سرد کن روشن است، حالات زیر را داریم:
\begin{latin}
    \centering
    \begin{tikzpicture}[->, >=stealth', auto, semithick, node distance=3.5cm]
    \tikzstyle{every state}=[fill=white,draw=black,thick,text=black,scale=1]
    \node[align=center,initial,state,initial text=] (0)  {Out};
    \node[align=center,state] (1)[right of=0] {RPS4};
    \node[align=center,state] (2)[right of=1] {RPS6};
    \node[align=center,state] (3)[right of=2] {RPS8};
    \path
    (0) edge[bend left, above] node{$T > 35 \degree C$} (1)
    (1) edge[bend left, below] node{$T < 25 \degree C$} (0)
    (1) edge[bend left, above] node{$T > 40 \degree C$} (2)
    (2) edge[bend left, below] node{$T < 35 \degree C$} (1)
    (2) edge[bend left, above] node{$T > 45 \degree C$} (3)
    (3) edge[bend left, below] node{$T < 40 \degree C$} (2)
    ;
    \end{tikzpicture}
\end{latin}
همچنین برای گرم کن با سه درجه گرما داریم:
\begin{latin}
    \centering
    \begin{tikzpicture}[->, >=stealth', auto, semithick, node distance=3.5cm]
    \tikzstyle{every state}=[fill=white,draw=black,thick,text=black,scale=1]
    \node[align=center,initial,state,initial text=] (0)  {Out};
    \node[align=center,state] (1)[right of=0] {Heater 1};
    \node[align=center,state] (2)[right of=1] {Heater 2};
    \node[align=center,state] (3)[right of=2] {Heater 3};
    \path
    (0) edge[bend left, above] node{$T < 15 \degree C$} (1)
    (1) edge[bend left, below] node{$T > 30 \degree C$} (0)
    (1) edge[bend left, above] node{$T < 10 \degree C$} (2)
    (2) edge[bend left, below] node{$T > 20 \degree C$} (1)
    (2) edge[bend left, above] node{$T < 0 \degree C$} (3)
    (3) edge[bend left, below] node{$T > 10 \degree C$} (2)
    ;
    \end{tikzpicture}
\end{latin}

حال به توضیح کد پایتون پیوست شده و شبیه ساز می‌پردازیم. در ابتدا تابعی تعریف می‌کنیم که عملا حکم
گرفتن دیتا از محیط را دارد. بدین صورت که بعد از ۱ ثانیه و با توجه به دمای قبلی محیط، دمای جدیدی را
تولید می‌کند. این دما را بین
10- تا 50 درجه نگه می‌داریم
که خیلی از
\lr{FSM}
دور نشویم.

سپس حالات و دمای اولیه را تعیین می‌کنیم. به عنوان مثال من دمای اولیه را ۳۰ درجه در نظر گرفتم. در ادامه در یک
حلقه‌ی بی‌نهایت می‌افتیم و منتظر می‌مانیم که کاربر
\lr{CTRL + C}
را فشار دهد که از برنامه خارج شود. در اینجا با توجه به حالت قبلی و دمای فعلی، حالت بعدی را
تشخیص می‌دهیم. برای اینکه برنامه‌ی ما شبیه کد وریلاگ باشد، از
\lr{next} و \lr{previous state}
استفاده می‌کنیم و در آخر حلقه آن‌ها را مقدار دهی می‌کنیم.

همچنان وقتی وارد یکی از حالات
\lr{heater on} یا \lr{cooler on}
می‌شویم اینقدر صبر می‌کنیم تا در حالت
\lr{out}
هر کدام از
\lr{sub state}های
آنها برویم.

\end{document}
